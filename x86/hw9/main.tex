\documentclass{article}
\usepackage[UTF8,space,hyperref]{ctex}
\usepackage{amsmath, amsthm, amssymb, bm, color, framed, graphicx, hyperref, mathrsfs, physics}
\usepackage{listings}
\title{汇编语言第九次作业}
\author{石曜铭}
\date{\today}

\lstset{
     columns = fixed,       
     basicstyle = \linespread{0.8} \ttfamily,             % 设置行距,字体
     numbers = left,                                      % 在左侧显示行号
     numberstyle = \tiny \color{gray},                    % 设定行号格式
     keywordstyle = \bfseries \color[RGB]{40,40,255},     % 设定关键字颜色
     numberstyle = \footnotesize \color{blue},           
     commentstyle = \color[RGB]{0,96,96},                 % 设置代码注释的格式
     stringstyle = \color[RGB]{128,0,0},                  % 设置字符串格式
     frame = single,                                      % 不显示背景边框
     backgroundcolor = \color[RGB]{245,245,244},          % 设定背景颜色
     showstringspaces = false,                            % 不显示字符串中的空格
     language=C                                           % 设置语言
}

\begin{document}

\maketitle

\textbf{3.58}

\begin{lstlisting}[language=C]
long decode2(long x, long y, long z) {
	y -= z;
	x *= y;
	long res = y;
	res >>= 63;
	res <<= 63;
	res ^= x;
	return res;
}
\end{lstlisting}

\textbf{3.59}

解释:首先将 $x, y$ 分别符号扩展至 $128$ 位,然后将 $x, y$ 分别表示成 $x = x_h \cdot 2^{64} + x_l, y = y_h \cdot 2^{64} + y_l$ 。

这样 
$$
x \cdot y = x_h\cdot y_h \cdot 2^{128} + (x_h\cdot y_l + x_l \cdot y_h) \cdot 2^{64} + x_l\cdot y_l.
$$

高于 $128$ 位的被忽略。实际上因为 $x_h, y_h$ 是符号位扩展的结果,所以结果只会是 $-1, 1$,而这实际上就是结果的符号位。

将结果 $p = x\cdot y$ 表示为 $p = p_h \cdot 2^{64} + p_l$ ,对比系数可得
$$
\begin{cases}
p_h &= x_h \cdot y_l + x_l \cdot y_h, \\ 
p_l &= x_l \cdot y_l.
\end{cases}
$$

代码首先取出 $x$ 的最高位,作位扩展和 $y$ 相乘,然后再求出 $x, y$ 低位相乘的结果。最后将乘法的高位和低位结果存储在 $\texttt{rdi}$ 所指向的内存地址。

\textbf{3.60}

(A) $\texttt{x}$ 在 $\texttt{rdi}$ 寄存器中,$\texttt{n}$ 在 $\texttt{esi}$ 寄存器中,$\texttt{result}$ 在 $\texttt{rax}$ 寄存器中,$\texttt{mask}$ 在 $\texttt{rdx}$ 寄存器中。

(B) $\texttt{result}$ 初值为 $0$,$\texttt{mask}$ 初值为 $1$ 。

(C) 不等于 $0$ 。

(D) $\texttt{mask} \leftarrow \texttt{mask} >> n$ ,其中 $n$ 只取低八位。

(E) $\texttt{result} \leftarrow \texttt{result} | \texttt{mask} \& \texttt{x}$ 。

(F) $0, 1, != 0, \texttt{mask} >> n, \texttt{x} \& \texttt{mask}$ 。

\textbf{3.61}

\begin{lstlisting}[language=C]
long cread_alt(long *xp) {
	long test = xp == 0;
	long result = 0; 
	if (!test) result = *xp;
	return result;
}
\end{lstlisting}

\textbf{3.62}

\begin{lstlisting}[language=C]
typedef enum { 
	MODE_A, MODE_B, MODE_C, MODE_D, MODE_E 
} mode_t;

long switch3(long *p1, long *p2, mode_t action) {
  long result = 0;
  switch(action) {
    case MODE_A:
      result = *p2;
      *p2 = *p1;
      break;
    case MODE_B:
      *p1 = *p1 + *p2;
      result = *p1;
      break;
    case MODE_C:
      *p1 = 59;
      result = *p2;
      break;
    case MODE_D:
      *p1 = *p2;
      result = 27;
      break;
    case MODE_E:
      result = 27;
      break;
    default:
      result = 12;
      break;
  }
  return result;
}
\end{lstlisting}

\textbf{3.63}

\begin{lstlisting}[language=C]
long switch_prob(long x, long n) {
  long result = x;
  switch(n) {
    case 60:
    case 62:
      result = x * 8;
      break;
    case 63:
      result = x >> 3;
      break;
    case 64:
      x = x << 4 - x;
    case 65:
      x = x * x;
    default:
      result = x + 0x4B;
  }
  return result;
}
\end{lstlisting}

\textbf{3.64}

(A) 设第一、二、三维大小分别为 $R, S, T$,设首地址为 $A$,每个元素的 $size$ 为 $L$,那么对元素 $(i, j, k)$,其地址为
$$
\texttt{dest} = A + L \times (i \times S \times T + j \times T + k).
$$

(B) 由代码易知,首先计算了 $j \times 13$ ,然后计算了 $i \times 65$,所以
$$
\begin{cases}
S \times T &= 65, \\ 
T &= 13, \\ 
8\times R \times S \times T &= 3640 .
\end{cases}
$$

所以 $R = 7, S = 5, T = 13$ 。


\textbf{3.65}

(A) $\texttt{rdx}$ 。

(B) $\texttt{rax}$ 。

(C) $M = 120 / 8 = 15.$

\texttt{3.66}

由代码可知,每次 $\texttt{rcx}$ 加的偏移量是 $\texttt{r8}$,而 $,\texttt{r8} = 8\times(4\times n + 1)$ ,所以 $NC(n) = 4\times n + 1$ 。又,跳出循环的条件是计数器 $\texttt{rdi} == \texttt{rdx}$,而 $\texttt{rdx} = 3\times n$ ,所以 $NR(n) = 3\times n$ 。

\texttt{3.67}

(A) $0 \sim 7: x; 8\sim 15: y; 16\sim 23: \&z; 24\sim 31: z.$

$\texttt{rsp} = 0, \texttt{rdi} = 64.$

(B) 传递了 s.a[0], s.a[1], s.p,以及一个新的地址 $\texttt{rdi} = \texttt{rsp} + 64$ 。

(C) 通过 $\texttt{rsp}$ 加偏移量。

(D) 以 $\texttt{rdi}$ 为起始地址。

(E) $0 \sim 7: x; 8\sim 15: y; 16\sim 23: \&z; 24\sim 31: z; 64\sim 71: y, 72\sim 80: x, 81\sim 87: z.$

(F) 由调用者开辟内存并把起始地址传给被调用者,然后被调用者在指定内存把结构体的内容保存,然后返回首地址。


\textbf{3.68}

由 $\texttt{movslq 8(\%rsi),\%rax}$,且 $t$ 是 $int$ 类型,$4$ 字节对齐,有 $4 < B \le 8$ 。

由 $\texttt{addq 32(\%rsi),\%rax}$,且 $u$ 是 $long$ 类型,$8$ 字节对齐,有 $24 < 12 + A\times 2 \le 32$ 。

由 $\texttt{movq \%rax,184(\%rdi)}$,且 $y$ 是 $long$ 类型,$8$ 字节对齐,有 $176 < A\times B\times 4 \le 184$ 。

综上,只有 $A = 9, B = 5$ 满足要求。

\end{document}