\documentclass{beamer}
\usefonttheme{professionalfonts}
\usepackage[UTF8,space,hyperref]{ctex}
\usepackage{graphics}
\usepackage{wrapfig}
\usepackage{ulem}
\usetheme{Madrid}
\usepackage{tikz}
\usecolortheme{default}
\usepackage{graphicx}
\usepackage{listings}

\title{从文本文件统计最高频单词}
\subtitle{汇编语言实验报告}
\author{石曜铭 钱翰林}
\institute{UCAS}
\date{\today}

\begin{document}

\begin{frame}
\titlepage
\end{frame}

\section{项目概述}
\begin{frame}{项目概述}
\begin{itemize}
\item 需求分析:
\begin{itemize}
\item 读取含英文/空格/标点的文本文件
\item 分割单词并统计频率
\item 输出最高频单词
\end{itemize}
\item 一些问题:
\begin{itemize}
\item 处理超大文件(超内存容量):有时文件大小会超过缓冲区大小
\item 优化统计时间复杂度:暴力匹配是 $O(n^2)$ 的,怎样优化
\item 跨缓冲区单词完整性的处理
\end{itemize}
% \item 语言:x86\_64 的 AT&T 语法
\end{itemize}
\end{frame}

\section{设计思路}

\begin{frame}{设计思路}

\textbf{分块读取策略}:
使用 10KB 大小的 buffer 对文件进行分块读取,分别处理每次读到 buffer 中的字符串。

\textbf{数据管道设计}:
\begin{itemize}
\item 主缓冲区 + 溢出缓冲区:使用 buffer + overflow\_buf 分别存储每次读到的内容和最后一个有效字符连续段的内容
\item 每次读 buffer 时先将上一轮中 overflow\_buf 内的内容置于 buffer 的头部,拼接成完整的读入。
\end{itemize}
\end{frame}

\begin{frame}{程序架构}

\begin{itemize}
\item 文件处理层 - 系统调用封装
\item 数据处理层 - 双缓冲区流水线
\item 算法层 - Trie树核心逻辑
\item 内存管理层 - 堆分配控制
\end{itemize}
\end{frame}

\section{关键技术实现}
\begin{frame}{跨缓冲区处理机制}
\begin{enumerate}
\item 逆向边界检测:从缓冲区尾部扫描,将最后一个有效字符连续段存到 Overflow\_buf 中。
\item 溢出缓冲区管理
\begin{itemize}
\item 100字节环形缓冲区
\item 首尾相接验证
\end{itemize}
\end{enumerate}
\end{frame}

\begin{frame}{Trie树的设计}
\begin{itemize}
\item Trie 树上每个 Node 保存 26 个指向子结点的指针,整数类型 Count 表示节点作为字符串末尾出现的次数,char 指针保存该节点存的字符串。
\item 具体地,我们为每个 Node 开辟 $8 \times 26 + 8 + 8 = 224$ 个 byte,在 .bss 段开辟 100000 个 Node 节点(可以开更多),作为 Trie 树所用空间。
\end{itemize}
\end{frame}

\section{调试心得}
\begin{frame}{调试心得}
在程序开发过程中,我们经历了数次棘手的调试挑战。我们通过 gdb 调试工具进行调试,设置断点,在重要的位置逐行运行,锁定 bug 根源所在。

以下列举一些在开发中发现的 bug :
\begin{itemize}
\item 在 process\_loop 中,调用的 is\_alpha 函数会将 rax 改为 $0$ 或 $1$,但调用接下来的 to\_lower 函数时没有将字符重新赋值给 rax;
\item 在 trie\_get\_max 中的一个分支忘记将 rcx 自增,导致死循环;
\item 在 memcpy 函数中,因对堆内存申请的不了解,写的代码实际上没有成功申请堆内存,导致 Segmentation fault;
\item ......
\end{itemize}

经过长时间不眠不休的调试,我们终于得到了一个相对完整的健壮的程序。
\end{frame}

\section{总结展望}
\begin{frame}{项目总结}
在本次大作业中,我们主要进行的设计有:
\begin{itemize}
\item 汇编级内存控制技术
\item 跨缓冲区单词的处理方案
\item 使用紧凑型 Trie 结构维护字典
\end{itemize}
\end{frame}

\begin{frame}{未来优化方向}
\begin{itemize}
\item 内存优化:为 Trie 树节点动态分配内存,而不是在 .bss 段预加载
\item 混合索引结构:
\begin{itemize}
\item 使用 Patricia Trie 来维护,效率更高
\item 概率跳表加速
\end{itemize}
\item 标准扩展:
\begin{itemize}
\item Unicode全字符支持
\item 分布式版本
\end{itemize}
\end{itemize}
\end{frame}

\end{document}