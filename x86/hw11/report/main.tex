\documentclass[12pt]{article}
\usepackage{geometry}
\usepackage{listings}
\usepackage{xcolor}
\usepackage[backref]{hyperref}
\usepackage{ctex}
\hypersetup{hidelinks}
\geometry{a4paper, margin=1in}

\definecolor{codebg}{rgb}{0.95,0.95,0.95}
\lstset{
  basicstyle=\ttfamily\footnotesize,
  backgroundcolor=\color{codebg},
  frame=single,
  breaklines=true,
  postbreak=\mbox{\textcolor{red}{$\hookrightarrow$}\space},
}

\title{SQLite \texttt{hwtime.h} 文件嵌入式汇编代码分析报告}
\author{石曜铭}
\date{\today}

\begin{document}

\maketitle

\section{引言}
本文以 SQLite 项目的 \texttt{src/hwtime.h} 文件为对象,对其中的嵌入式汇编代码进行转换、注释及分析。

hwtime.h 文件的作用是为不同平台(主要是x86、x86\_64、PowerPC 架构)提供一种高精度计时功能,获取CPU级别的高分辨率时钟计数,常用于性能分析和代码剖析。文件内定义了 sqlite3Hwtime() 函数,它直接读取CPU的时间戳计数器(如x86的RDTSC指令),返回自处理器启动以来的时钟周期数。这可以用来测量代码片段实际消耗的CPU周期,精度远高于普通的系统时间函数。为了适应不同的编译器和CPU架构,可以使用内联汇编方便地实现支持。

\section{x86/x86\_64 架构的实现}

\subsection{GCC 下的 x86/x86\_64 实现}

\textbf{C 代码片段:}
\begin{lstlisting}[language=C]
__inline__ sqlite_uint64 sqlite3Hwtime(void){
   unsigned int lo, hi;
   __asm__ __volatile__ ("rdtsc" : "=a" (lo), "=d" (hi));
   return (sqlite_uint64)hi << 32 | lo;
}
\end{lstlisting}

\textbf{转换为汇编(Intel 语法):}
\begin{lstlisting}[language={[x86masm]Assembler}]
rdtsc                   ; 读取时间戳计数器到 EDX:EAX
; EAX -> lo
; EDX -> hi
; (C 端合并为 64 位整数返回)
\end{lstlisting}

\textbf{分析与注释:}
\begin{itemize}
  \item \texttt{rdtsc} 指令将 CPU 自上电以来的时钟周期数分别放入 EDX:EAX。
  \item C 代码通过内联汇编提取这两个寄存器的值,组合成 64 位整数,便于高精度计时。
  \item 用于性能分析和微基准测试。
\end{itemize}

\subsection{MSVC 下的 x86 实现}

\textbf{C 代码片段:}
\begin{lstlisting}[language=C]
__declspec(naked) __inline sqlite_uint64 __cdecl sqlite3Hwtime(void){
   __asm {
      rdtsc
      ret
   }
}
\end{lstlisting}

\textbf{转换为汇编(Intel 语法):}
\begin{lstlisting}[language={[x86masm]Assembler}]
rdtsc                   ; 读取时间戳计数器到 EDX:EAX
ret                     ; 直接返回,返回值即为 EDX:EAX 组成的 64 位整数
\end{lstlisting}

\textbf{分析与注释:}
\begin{itemize}
  \item \texttt{\_\_declspec(naked)} 表示该函数无自动生成的函数序言/结尾,直接运行汇编体。
  \item 与 GCC 版本功能一致,返回硬件周期计数。
\end{itemize}

\section{PowerPC 架构的实现}

\textbf{C 代码片段:}
\begin{lstlisting}[language=C]
__inline__ sqlite_uint64 sqlite3Hwtime(void){
    unsigned long long retval;
    unsigned long junk;
    __asm__ __volatile__ ("\n\
        1:      mftbu   %1\n\
                mftb    %L0\n\
                mftbu   %0\n\
                cmpw    %0,%1\n\
                bne     1b"
                : "=r" (retval), "=r" (junk));
    return retval;
}
\end{lstlisting}

\textbf{转换为汇编(PowerPC 语法):}
\begin{lstlisting}[language={[x86masm]Assembler}]
1:
    mftbu   rJunk      ; 读取 time base upper (高 32 位) 到 rJunk
    mftb    rRetLow    ; 读取 time base lower (低 32 位) 到 rRetLow
    mftbu   rRetHi     ; 再次读取 time base upper 到 rRetHi
    cmpw    rRetHi, rJunk ; 比较两次高位是否一致
    bne     1b         ; 若不一致,表明期间发生溢出,重试
    ; rRetHi:rRetLow 即为完整的 64 位时间基准
\end{lstlisting}

\textbf{分析与注释:}
\begin{itemize}
    \item PowerPC 架构的时间基准寄存器(time base)分高低两部分读取。
    \item 为避免低 32 位读取时发生溢出,采用两次读取高位,对比防止跨界,保证原子性。
\end{itemize}

\section{无法使用汇编的情形}

\textbf{C 代码片段:}
\begin{lstlisting}[language=C]
sqlite_uint64 sqlite3Hwtime(void){ return ((sqlite_uint64)0); }
\end{lstlisting}

\textbf{分析:}
\begin{itemize}
    \item 若编译器/平台不支持内联汇编,则直接返回 0,不提供高精度计时功能,保证代码跨平台可编译。
    \item 该函数一般只用于调试和性能分析,不影响 SQLite 主功能。
\end{itemize}

\section{综合分析}

\begin{itemize}
    \item \texttt{hwtime.h} 通过内嵌汇编,跨平台地实现了硬件级高分辨率计时功能,便于开发者进行微秒级性能分析。
    \item 针对主流 CPU 架构(x86, x86\_64, PowerPC)分别优化实现,保证效率与准确性。
    \item 若平台不支持,则以安全的空实现兜底,提升兼容性。
\end{itemize}

\section{参考资料}
\begin{itemize}
    \item \href{https://www.intel.com/content/www/us/en/developer/articles/technical/intel-sdm.html}{RDTSC 指令 - Intel® 64 and IA-32 Architectures Software Developer’s Manual}
    \item \href{https://www.nxp.com/docs/en/application-note/AN3530.pdf}{PowerPC Time Base Register}
    \item SQLite 官方源码仓库 \url{https://github.com/sqlite/sqlite/blob/master/src/hwtime.h}
\end{itemize}

\end{document}